
%%%%%%%%%%%%%%%%%%%%%%%%%%%%%%%%%%%%%%%%%%%%%%%%%%%%%%%%%%%%%%%%%%%%%%%%%%%%
% This will help you in writing your homebook
% Remember that the character % is a comment in latex

% Divide the work you have done in each of the chapters used 
% during the lab lessons in a new chapter, as in the example below, 
% using a coherent title 


% For each chapter you can include :

%-----------------------------
% VHDL file, using the sintax:


	%\begin{listato}
	%\lstinputlisting{./exeMPLE/listato1.vhd}
	%\end{listato}

% the path to the file must be correct, obviously
% Should you have listings written in other languages the method is the 
% same, but the language set up must be changed using a different 
% setting for the command \lstset{language=VHDL} in file homebook.tex 


%-----------------------------
% figures in postcript (ps) or encapsulated postcript (eps)
% format, using the syntax:

%	\begin{figure}[h]
%	\centering
%	\includegraphics[width=9cm]{./cap1/figure1.eps}
%	\caption{Put a caption if you want (didascalia...:)))}
%	\label{put-a-label-for-referring-to-this-picture}	
%	\end{figure}

% the path to the file must be correct, obviously
% you can refer to this picture in any point of your document
% by typing the instruction:

% 	\ref{put-a-label-for-referring-to-this-picture}

% that is using the same label you put in the fiure label
% when you will run the "latex command" an automatic reference to
% this figure with the correct enumeration will be inserted


%-----------------------
% comment in text format (if you are not skilled in latex and don't want to be)
% using the sintax:

	%\begin{verbatim}
	% blablabla 
	%\end{verbatim}

% The verbatimg includes text as it is, as you could write in a normal text file 

% (BETTER) If uou want to write enhancing all the latex possibilities you 
% should add to you text a few commands in some particular cases. 
% In the following you have and example of a few chapters roughtly commented
% and written all in this file: remember that you can saparate
% each chapter in different files (this is always what a latex pro does) 
% and include them using the instruction: \input{./directoryxx/fileyy.tex}


%%%%%%%%%%%%%%%%%%%%%%%%%%%%%%%%%%%%%%%%%%%%%%%%%%%%%%%%%%%%%%%%%%%%%%%%%%%%%%%
%%%%%%%%%%%%%%%%%%%%%%%%%%%%%%%%%%%%%%%%%%%%%%%%%%%%%%%%%%%%%%%%%%%%%%%%%%%%%%%%%
%%%%%%%%%%%%%%%%%%%%%%%%%%%%%%
% Beginning of latex commands
% You can copy this in a new file (e.g. cap1/cap1.tex) and include it here
% using the command : \input{./cap1/cap1.tex}


\chapter[Overview and Features]{Overview and Features}
Our DLX processor is a 5-stage pipelined in-order microprocessor which fully supports
general integer instructions and some long latency special integer instructions,
like signed and unsigned multiplication and division and unsigned square root.

With maximum forwarding technology, the processor only suffers 1 type of logic hazard; 
Read After Load (RAL).

In order to eliminate or reduce the delay caused by branching, the Branch unit is organized
in the 2nd stage (ID), which allows it to make the decision as soon as possible.
Furthermore, a 2-bit-saturating-counter-based Branch Prediction unit is embedded in the Branch unit and is used to
predict the result of branch in case of a Branch After Load (BAL) stall.

\chapter[Structure]{Structure}
The whole DLX microprocessor is organized in 4 parts; Control unit, Datapath,
the instruction RAM and the data RAM. As shown in figure \ref{fig:Dlx}

\section[Control Unit]{Control Unit}
Control Unit mainly produces the control words to control the behavior of our microprocessor's Datapath.
It also needs to decide how a stall signal affects the pipeline and whether a branch
is taken or not.

Our microprocessor's Control Unit consists of a Control word generator, a Stall generator and a Branch unit.
As shown in figure \ref{fig:ControlUnit}

\subsection[Control Word Generator]{Control Word Generator}
The Control word generator (CwGenerator) is a tiny-micro-processor-based module with 2 memory levels;
a relocated memory and a microcode memory.

It first looks up a signal called {\it upc2} corresponding to the current instruction in the {\it stage 2 (ID)} and then gets the control
words from the microcode memory based on this {\it upc2} signal. The {\it upc2} can be delayed with offset in each clock cycle,
which makes it {\it upc3}, {\it upc4} and {\it upc5}.

For R-type and F-type instructions, all the control words are the same
except for the part controlling the execution units (ALU, MUL, DIV).
So a single control word that consists of many bits is introduced to solve this sharing problem among R-type
and F-type instructions.

At the end of process, depending on {\it branch flag} and {\it stall flag}, the control words will be masked in
a specific way to form the final control words.

The structure of Control word generator is shown in figure \ref{fig:CwGenerator}

\subsection[Stall Generator]{Stall Generator}
The Stall Generator deals with different signals which will lead to stall any stage
of the pipeline. The signals include:

\begin{description}
  \item[RAL] Read After Load, 3rd stage (EXE).
  \item[BPW] Branch Predict Wrong, 3rd stage (EXE).
  \item[JRAL] JR After Load, 2nd stage (ID).
  \item[MUL] Multiplication, 3rd stage (EXE).
  \item[DIV] Division, 3rd stage (EXE).
  \item[SQRT] Square Root, 3rd stage (EXE).
\end{description}

The detail of hazard handling will be covered in Section \ref{ch:hazard}.

For each kind of hazard, we use a F.S.M. and a logic block to determine the behavior of
Datapath, the former is for following clock cycles and the latter is for the current clock
cycle. Finally, the {\it stall\_flag} signal is produced by performing OR operation between
the resulting signals obtained from these blocks.

The structure of Stall generator is shown in figure \ref{fig:StallGenerator}

\subsection[Branch]{Branch}
The Branch Unit calculates the result of branch instructions by checking the value
of the involving register. It will make predictions when the value of this register
is not ready.

The detail of Branch After Load (BAL) hazard will be covered in section \ref{ch:hazard}. And the
detail of Branch Prediction will be covered in section \ref{ch:bpu}

The structure of Stall generator is shown in figure \ref{fig:StallGenerator}

\section[Datapath]{Datapath}
The Datapath contains the basic 5 stages and one additional stage (EXTRA); it connects to Instruction RAM
in stage 1 and 2; data RAM in stage 4.

\subsection[IF (Stage 1)]{IF (Stage 1)}
Instruction Fetch sends PC to Instruction RAM and calculates NPC. As shown in figure \ref{fig:if}.

\subsection[ID (Stage 2)]{ID (Stage 2)}
Instruction Decode receives an instruction from Instruction RAM and depending on its type, the processor in this stage
reads operands from Register File, extends the immediate value and calculates the PC for the next instruction.

The Register File works at the falling edge of the clock signal. It has 2 read ports and 1 write port. There are 32 4-byte integer registers
inside the data area.

The Extend Unit works in both signed and unsigned modes.

The address which is not selected by the branch prediction unit is kept and delayed in case of a wrong prediction occurs,
if so, this address could be restored.

The structure is shown in figure \ref{fig:id}

\subsection[EXE (Stage 3)]{EXE (Stage 3)}
Execution stage receives operands and control words to perform different types of operations. For the ALU, all the operations
are completed within 1 clock cycle. For MUL and DIV, the operations will take more than 1 clock cycle to be completed. The long
latency operations will be covered in section \ref{ch:lli}

\subsection[MEM (Stage 4)]{MEM (Stage 4)}
Memory stage receives the resulting signals from the operations performed in the previous stage and based on the given instruction
the it will either store the resulting value in Data RAM or bring a value needed from Data RAM.

\subsection[WB (Stage 5)]{WB (Stage 5)}
WriteBack stage takes the final value given by the Memory stage and send it to the Register File (RF). An important point here is
that the address in which this value is going to be set has been forwarded from stage 2 (ID).

\subsection[EXTRA (Stage 6)]{EXTRA (Stage 6)}
Extra stage will basically take the same output of stage 5 and send it back to the memory stage in orded to handle hazards. this will be explained in details in the section… 

